\begin{frame}{Compatibilité}

  \begin{block}{Langage utilisateur / Format}
    \begin{itemize}
    \item Le public ciblé est très précis.
    \item Les termes le sont donc aussi.
    \item<pro@1-> Les formats sont standard (sortie ``dot'' de
      l'automate).
    \end{itemize}
  \end{block}


  \begin{block}{Organisation des informations}
    \begin{itemize}
    \item<pro@1-> L'information nécessaire (comment taper et où) est tout en haut.
    \item<con@1-> Le résultat s'affiche en bas de la page, et il est
      nécessaire de scroller.
    \item Pour aller plus en profondeur dans les choix, il faut lire
      le milieu de la page.
    \end{itemize}
  \end{block}


\end{frame}
