\begin{frame}{Guidage}

  \begin{block}{Incitation}
    \begin{itemize}
    \item<pro@1-> La page contient le format des formules acceptées et
      comment les écrire.
    \item<pro@1-> Les zones colorées correspondent à des zones
      interactives.
    \item Une seule page, découpée avec chaque partie disposant de son
      propre titre.
    \end{itemize}
  \end{block}

  \begin{block}{Groupement/Distinction par la localisation}
    \begin{itemize}
    \item<pro@1-> Chaque item est inclus dans un groupe clair et titré en
      fonction de son appartenance (par exemple ``Translator
      Algorithm'').
    \item<pro@1-> Chaque titre a un format spécifique (coloré et
      cliquable).
    \end{itemize}
  \end{block}
\end{frame}

\begin{frame}{Guidage (suite)}
  \begin{block}{Feedbacks}
    \begin{itemize}
    \item Les informations saisies par l'utilisateur restent affichées.
    \item<con@1-> L'affichage du résultat n'est pas assez mis en avant.
    \end{itemize}
  \end{block}

  \begin{block}{Lisibilité}
    \begin{itemize}
    \item<pro@1-> La façon dont on doit écrire les propositions est
      clairement expliquée.
    \item Un tableau rappelle les différents opérateurs.
    \item Les titres ne sont pas (généralement) centrés.
    \item La police est droite.
    \item Lettres sombres sur fond clair.
    \end{itemize}
  \end{block}
\end{frame}
