\documentclass{beamer}

\usepackage[frenchb]{babel}
\usepackage[utf8]{inputenc}
\usepackage[T1]{fontenc}

\usetheme{Copenhagen}
\addtobeamertemplate{footline}{\insertframenumber/\inserttotalframenumber}


\author {$\{CSI\} \setminus \{\text{Laurent Senta}\}$}
\date\today
\title[Ergonomie: Spot online ]{Présentation Ergonomie et Interactions Humains-Technologie : \og Spot online translator\fg}

\institute{Epita - CSI}
\begin{document}

\begin{frame}
  \maketitle
\end{frame}


\begin{frame}{Présentation}
  \begin{block}{Spot}
    \begin{itemize}
    \item Projet développé au LRDE.
    \item Bibliothèque de Model-Checking en C++.
    \item Un des points forts: Traduction d'un certain type de formules
      en automates.
    \end{itemize}
  \end{block}

  \begin{block}{Interface web}
    \begin{itemize}
    \item But : Rendre facile l'utilisation de Spot pour traduire en
      automate.
    \item Lien : \url{http://spot.lip6.fr/ltl2tgba.html}
    \end{itemize}
  \end{block}
\end{frame}



\begin{frame}
  \frametitle{Guidage}
  ...
\end{frame}

\begin{frame}{Contrôle de travail - Brièveté}
  \begin{block}{Concision}
    \begin{itemize}
    \item Une ligne à remplir.
    \item Beaucoup d'informations, pas nécessaires.
    \end{itemize}
  \end{block}

  \begin{block}{Actions minimales}
    \begin{itemize}
    \item Configuration minimale, une formule à entrer.
    \item Configuration maximales: Sélection parmi des boutons clairs.
    \end{itemize}
  \end{block}

  \begin{block}{Densité informationnelle}
    \begin{itemize}
    \item Beaucoup d'informations à l'écran.
    \item Possibilité de conserver que ce qui nous intéresse.
    \end{itemize}
  \end{block}
\end{frame}

\begin{frame}{Contrôle explicite}

  \begin{block}{Actions explicites}
    \begin{itemize}
    \item La ligne d'entrée permet d'enregistrer une formule et
      l'action est confirmée par ``Send''.
    \item La clarté de ce mot est discutable.
    \end{itemize}
  \end{block}

  \begin{block}{Contrôle utilisateur.}
    \begin{itemize}
    \item L'utilisateur n'a pas de contrôle, cela n'a pas de sens ici.
    \end{itemize}
  \end{block}

\end{frame}

\begin{frame}
  \frametitle{Adaptabilité}
  ...
\end{frame}

\begin{frame}{Gestion des erreurs}
  \begin{block}{Protection}
    \begin{itemize}
    \item<pro@1-> Liste des options possibles pour les algorithmes.
    \item Texte explicatif de la syntaxe des formules.
    \end{itemize}
  \end{block}

  \begin{block}{Messages d'erreur}
    \begin{itemize}
    \item Sortie d'erreur de l'interpréteur.
    \item Neutre.
    \item<con@1-> Mais parfois peu clair.
    \end{itemize}
  \end{block}

  \begin{block}{Correction}
    \begin{itemize}
    \item Il y a une explication des erreurs.
    \item pas d'annulation de commande ni de retour en arrière.
    \end{itemize}
  \end{block}
\end{frame}

\begin{frame}{homogenéité}
  \begin{block}{Procédures similaires}
    \begin{itemize}
    \item Chaque option est cliquable de la même manière.
    \item Lorsqu'il y a une grande différence, des onglets sont créés.
    \end{itemize}
  \end{block}
  \begin{block}{Localisation des fenêtres}
    Etant donné qu'il n'y a qu'une seule fenêtre, les autres points
    n'ont pas de sens.
  \end{block}

\end{frame}

\begin{frame}{Significiance des codes}
  \begin{block}{Titres}
    Les titres correspondent exactement à ce que contient la section.
  \end{block}

  \begin{block}{Abréviations}
    Il n'y en a pas.
  \end{block}
\end{frame}

\begin{frame}
  \frametitle{Compatibilité}
  ...
\end{frame}


\begin{frame}{Conclusion}
  \begin{block}{Points négatifs}
    \begin{itemize}
    \item Trop d'informations,
    \item Facile de ne pas voir qu'il y a eu un résultat,
    \item Les erreurs ne sont pas assez adaptées à un non-initié.
    \end{itemize}
  \end{block}

  \begin{block}{Points positifs}
    \begin{itemize}
    \item Options par défaut très bonnes.
    \item Possibilité de ne conserver que les informations importantes.
    \item Rappels de la syntaxe.
    \item Possibilité d'avoir la sortie ``dot''.
    \end{itemize}
  \end{block}
\end{frame}

\begin{frame}{Questions ?}

  Merci de votre attention,

  Avez vous des questions ?
\end{frame}

\end{document}
